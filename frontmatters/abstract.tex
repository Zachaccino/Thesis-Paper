\documentclass[../thesis.tex]{subfiles}

\begin{document}
\chapter*{Abstract}

The purpose of this thesis is designing a real-time communication system for digital power systems such as a solar farm to monitor and control the devices within them. The proposed system uses the LTE network as the means of communication for individual power generation device to contact the server hosted on the internet. In the proposed system, socket and message queue are used to achieve efficient real-time data visualisation over a web client. While a typical HTTP server is used to handle general requests. Certain time-consuming requests are handled asynchronously using job queue and workers. The performance of the system is measured by benchmarking against a simulated load under different system configuration and amount of load. The usability of the system is tested using an experimental solar panel setup and remotely monitoring the panel over a long period. The benchmarking result shows the proposed system is capable of processing up to 1000 devices accessing the system simultaneously at the rate of one request per device per second, running within a computer system with configurations that are comparable to a high-end consumer-grade computer. The usability testing also shows the system is capable of showing historical data, showing data update in real-time, and control the power generation device remotely. Therefore, we conclude the proposed system is expected to be capable of monitoring and controlling a large scale power system in real-time if enough computational resources are provided.

\end{document}