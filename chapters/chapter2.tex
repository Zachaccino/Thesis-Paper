\documentclass[../thesis.tex]{subfiles}

\begin{document}
\chapter{Literature Review}
\label{chap:litrew}


\section{Overview}
There are many existing solutions and work done by researchers exploring the ways to construct a communication system for monitoring and controlling of solar panels. They can be separated into three categories:

\begin{itemize}
\item Wired communication
\item Long-range wireless communication
\item Short-range wireless communication
\end{itemize}

The wired communication requires physical cables to connect the components of the system, the long-range wireless communication employs cellular network to transmit data over long distances wirelessly, and short-range wireless communication utilises ZigBee to form a mesh network which covers an arbitrary area.

While many pieces of literature explored different means of communication for the solar power generation system, only a few of them address how the data from solar panels can be processed more timely and efficiently. Related work in other similar fields had been reviewed which suggests the use of message queues and real-time distributed computing systems may improve the performance and usability of such a system.


\section{Wired communication}

The conventional method of monitoring a large scale solar farm is using physical cables. In this setup, a set of power converters read the current and voltage, then transmit the sensing data to the monitoring station over some physical cables. One of the benefits of using physical cables is its proven reliability that had been demonstrated throughout solar power generation systems around the world. Furthermore, physical cables have high bandwidth which enables complicated sensing data with high sampling rate can be transmitted without fully saturating the link \cite{SHARIFF20151730}. However, the researchers have found the lifespan of the system may be reduced due to physical cables are exposed to constant sunlight and rain \cite{SHARIFF20151730}. Furthermore, the signals in the cables may attenuate over long distances and countermeasures such as repeater may be required.


\section{Long range wireless communication}

One of the long-range wireless communication that had been studied is a GSM-based communication system for the solar-powered street light. In this system, each street light has a microcontroller monitoring the power generation, battery status, battery behaviours, and light behaviours. Each microcontroller communicates with a remote server through Short Message Service (SMS), which is the underlying technology that enables a mobile phone to send and receive text messages from or to other mobile phones. The SMS messages carry the sensing data from each street light to the remote server through an SMS gateway to be processed and stored in the database. Finally, a web console can be used to visualise the data \cite{SiregarSimon2014Spab}. The use of SMS solves a big issue that other means of communication such as Bluetooth, ZigBee, and Wi-Fi has, the lack of range. With this setup, a large area of street lights can be connected to the SMS gateway thanks to the large communication range of the underlying GSM communication system. However, the major drawback of using SMS as the underlying means of communication is the size of each SMS message must be within 160 characters \cite{etsi.org_1995}. The limitation means the sensing data cannot be very complicated and detailed, which limits the capability of the system and its use cases. The remote server also plays an important role in the communication system. It is composed of a GSM gateway that is connected to a web server, and the webserver is exposed to the internet so everyone can access the visualisation of the data over the internet. The web server is also connected to a database such that the sensing data is persistent even if the webserver is down \cite{SiregarSimon2014Spab}. Unfortunately, the cost estimate and the performance of such a system, and the underlying architecture of the web server are not documented in the literature. It is hard to quantitatively compare this solution to other solutions in terms of price and performance. This system is also intended to be used with street lights that have a much lower density than solar panels, and the behaviour of such a system is undocumented as the number of microcontroller increases. Therefore, it is uncertain that the GSM-based communication system would adapt well to a more demanding situation such as a solar farm.

Another long-range wireless communication system is a GPRS based communication system for solar panel monitoring and controlling. The system is designed around the concept of the Internet of Things (IoT) in mind and it is separated into three layers, the sensing layer, network layer, and application layer. The sensing layer contains sensors monitoring the characteristics of the solar panel. Then, the sensing data are fetched into a microcontroller equipped with a GPRS modem. Finally, the sensing data are sent to the internet through the GPRS modem. The application layer contains advanced functionalities such as data analytics, fault monitoring, generation monitoring, and functionalities that leverage the powerful processing capabilities of the server. Between the sensing layer and application layer, the network layer bridge the two layers by providing internet access and hosting database for persistent data storage \cite{AdhyaSoham2016AIbs}. The system utilises the GPRS cellular network, which is the primary means of communication between mobile phones in the 2000s. This technology has the benefit of covering a wide area and offers internet access with limited speed. That means a single GPRS cellular tower can cover significantly more solar panels comparing to other wireless technologies such as Bluetooth, ZigBee and Wi-Fi. Having direct internet access also means it doesn't have to translate the sensing data from one communication system to another like the GSM-based system, where data are being sent through SMS messages and translated at the GSM gateway into HTTP messages. Unfortunately, GPRS is still very slow by today's standards with only 14kbps upload \cite{3gpp.org}. It imposes a substantial limit on the complexity and amount of data being sent to the server for processing. More importantly, countries worldwide had been planning on decommissioning this old means of communication before the mid-2020s. Therefore a new carrier for the sensing data needs to be used. Similar to the previous literature, there is no cost estimation of such a system and no performance details were documented.


\section{Short range wireless communication}

A short-range communication method that is evolving around the IoT concept is ZigBee which is explored over the recent years. Like many other communication systems, there needs to be a ZigBee gateway with internet access to forwarding data from the solar panels to the server and persistent data store using HTTP protocols. Furthermore, the limited communication range of ZigBee allows power consumption to be is very low. The researchers claim a ZigBee module can be powered by non-rechargeable battery for two to three years and it can connect up to 65000 ZigBee modules \cite{SHARIFF20151730}. Since ZigBee is designed for modern IoT applications, it has more than sufficient bandwidth of up to 250kbps for complicated data exchange over the network. The problem with ZigBee is its limited range of around 1500 meters with a direct line of sight and the range may be reduced significantly if there are obstructions \cite{SHARIFF20151730}. To combat this issue, ZigBee modules can form a cluster of connected ZigBee network which allows it to cover a much wider area. The supported network topologies of ZigBee modules are Star, Cluster Tree, and Mesh Network \cite{SHARIFF20151730}.

The star topology has a primary ZigBee module that is directly connected to all secondary ZigBee modules. Since they are directly connected to the primary module, the resources of the secondary modules such as bandwidth are completely reserved for its communication with minimal latency. This topology doesn't increase the coverage of a ZigBee Network but it is suitable for a small cluster of solar panels to communicate with the gateway that has internet access such as rooftop solar systems \cite{SHARIFF20151730}.

The cluster tree topology is formed by connecting many networks with star topology. The sensing data from each module are sent and forwarded to the gateway through an arbitrary number of ZigBee modules assuming there exists at least one gateway within the network that is directly or indirectly reachable from the originating module. This topology enables arbitrary coverage of the communication system. The problem with this topology is some ZigBee modules that need to forward many other modules' sensing data may easily overwhelm its available bandwidth, leading to performance degradation \cite{SHARIFF20151730}.

Finally, the mesh topology allows each ZigBee module having more than two directly connected modules, which allows them to form a mesh network. Similar to the cluster tree topology where the coverage is arbitrary, it also has the benefit of reducing the potential to reach bottleneck as the sensing data can be sent through many different path \cite{SHARIFF20151730}.


\section{Computation system}

Since the lack of studies with regards to the computer system architecture used in the communication system for solar panels. System architectures for communication systems that require similar capabilities are reviewed instead. One of the proposed systems is designed for monitoring automotive manufacturing processes using IoT devices, and it incorporates Kafka, Apache Storm, and MongoDB to process monitor, and store data in real-time \cite{Syafrudin_2018}. In this architecture, Kafka is being used as a message queue for asynchronous communication between components of the system and it employs the publisher-subscriber design pattern. The pattern separates the responsibility of the system into producer and consumer where a producer generates data, assign it with a topic that consumer can subscribe to, and consumers are notified when a new piece of data from their subscribed topics is available. The Apache Storm is used for real-time processing. It is subscribed to the topics in the Kafka and reacts to those new data depending on the type of sensor. In general, it stores the new data into MongoDB, a persistent data storage, and publish new data to the user so data is available as soon as it is generated and processed. The advantage of this architecture is the user can view the data in real-time whenever it is ready, whereas many other simple architectures consist of a single HTTP web server requires the user to refresh the website constantly to download new data. This enables the user to make timely decisions and reduces the workload of the server as it only processes the data when it is available and the user doesn't query for data constantly.


\section{The missing piece of the puzzle}

Currently, most literature has been strongly focusing on the means of communication between the solar panels and gateway, or the computation systems that are processing the data in the background. There is barely any work was done that investigates the union of both worlds, that is, investigating a real-time computation system that maximises the benefits of the selected means of communication for the solar power generation systems. In my opinion, this is one of the most important pieces of the puzzle that is missing within this problem space because:

\begin{itemize}
\item Lack of documentation regarding the performance meaning we cannot quantitatively compare, justify, and identify the areas of improvement.
\item Lack of integration between the two research areas meaning we cannot predict if the selected combo would work together efficiently no matter how promising they are in its own right.
\item Lack of cost estimate of the system means we have no way to tell if the proposed solution is practical and feasible.
\end{itemize}



\end{document}