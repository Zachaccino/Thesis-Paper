\documentclass[../thesis.tex]{subfiles}

\begin{document}
\chapter{Introduction}

The rise of modern societies relies on electricity. There are countless innovations from sophisticated space stations travelling at thousands of kilometres per second to tiny little phones that you use all day every day, are powered by electricity. Needless to say, electricity is an integrated part of our modern societies.

Nowadays, the generation of electricity comes from two types of power plant, the non-renewable ones burn fossil fuels, and the renewable ones harvest energy from sunlight and wind\cite{Elecgen}. Since the non-renewable power plants are a big contributor to global warming, countries worldwide had been transitioning into harvesting more and more renewable energy in an attempt to slow down the global warming\cite{GrowingSolar}. As a result, larger and larger solar farms are being constructed, and more and more homes are opting to install solar panels on their rooftop.

As the size of solar power plants grows, the management difficulties and costs also grow with the size. The traditional way of monitoring and controlling solar power plants is using physical cables that are connected to each panel or converter, and they are exposed to constant punishment from harsh environments such as extreme heat and strong sunlight\cite{SHARIFF20151730}. This approach works well on small scale solar power plants, since the area is small and maintaining the physical cables are not a big problem. However, when the size of the solar power plant is a few kilometres across, the physical cables can be expensive to install and maintain\cite{SHARIFF20151730}.

A few solutions had been proposed that uses wireless communication to replace physical cables. ZigBee is one of the proposed solutions which uses point to point communication to form a mesh network that allows every solar panel within the network to communicate with each other and send data to the central controller\cite{SHARIFF20151730}. Another technology that is often being mentioned in these solutions is the cellular network. These systems employ GPRS or SMS to transfer messages to the cellular base station, and then the data is sent to the monitoring station over the internet\cite{AdhyaSoham2016AIbs,SiregarSimon2014Spab}. However, these solutions either have a very limited range like the ZigBee solution, or the uplink and downlink bandwidth is too slow for complicated data recording tasks such as the GRPRS and SMS solution. Furthermore, the GPRS network is being decommissioned around the globe and it will not be viable soon.

Over the years, the cost of cellular communication modules has decreased while the speed, bandwidth, and capacity had increased significantly\cite{CellGenComp}. Currently, the LTE cellular network, also known as 4G, offers a long-range communication method that has great uplink and downlink bandwidth. This is great for large scale power systems because just a handful of LTE base stations can cover the entire large scale solar power plant. The LTE network having high bandwidth also means complicated monitoring and controlling can be achieved. Additionally, the LTE module is relatively cheap to be incorporated in the microcontroller nowadays. These are the major reasons that LTE had been chosen as the method of communication for the monitoring and controlling system that this thesis proposes.

Interestingly, many research papers had extensive research into the feasibility and usability of different means of communication such as cellular network and ZigBee. Only a handful of them discusses the software such as the system architecture they used to achieve their results. Let along quantitatively represents the performance of such a system. This is also one of the goals of this thesis, which is deep dives into the system architecture and quantitatively measure the performance of the whole communication system to set the benchmark for such a system for future comparison.

When designing such a monitoring and controlling system, the characteristics of renewable power plants need to be considered. The renewable power plants are usually highly dynamic and not reliable, that is, the power generated by these plants can fluctuate from high to low depending on the weather or the time of day, and the power generation doesn't follow the actual electricity demand. That means the system must provide timely feedback to the operators for making decisions and ways to control the power generation devices to react to the changes. Therefore, the system proposed in this thesis must be capable of doing so.

This thesis is structured as follows. The literature review in \autoref{chap:litrew} includes a discussion of existing solutions. Then, an overview of the proposed system on a very high level is provided in \autoref{chap:syso} for readers to gain a basic understanding of the components in the system and their responsibilities on a high level. \autoref{chap:arch} take a deep dive into the architecture and behaviours of the proposed system. The quantitative analysis includes the performance of the system is presented in \autoref{chap:bm}. The real-world usability testing is detailed in the \autoref{chap:rw}. Finally, the conclusions of the thesis are presented in \autoref{chap:conclu}.
\end{document}